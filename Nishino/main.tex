%%%%%%%%%%%%%%%%%%%%%%%%%%%%%%%%%%%%%%%%%%%%%%%%%%%%%%%%%%%%%%%%%%%%%%%%%%%%%%
%%
%%
%%	奈良工業高等専門学校 情報工学科
%%
%%	第 4 学年 情報工学実験 III
%%
%%		「コンピュータネットワークに関する実験」実験報告書サンプル
%%
%%				平成 28 年  5 月 26 日(pLaTeX 版)第 1 版
%%
%%			西野 貴之(nishino@info.nara-k.ac.jp)
%%
%%
%%%%%%%%%%%%%%%%%%%%%%%%%%%%%%%%%%%%%%%%%%%%%%%%%%%%%%%%%%%%%%%%%%%%%%%%%%%%%%

%%
%% ★ プリアンブルの宣言
%%
%%	プリアンブル部は,文書全体にわたって有効な設定を行う部分です.
%%	プリアンブル部で設定を行う主な項目には,以下に示すようなものが
%%	あります.
%%

%%
%%	(1)ドキュメントクラスの設定
%%
%%	    ドキュメントクラスの宣言では,文書全体の形式について記述します.
%%
%%	    書式:\documentclass[スタイルオプション]{文書スタイル}
%%
%%	    以下に「スタイルオプション」や「文書スタイル」について説明します.
%%
%%	    ◎ 文書スタイル
%%	       文書スタイルでは,LaTeX 文書全体の形式(書籍や論文,レポート
%%	       など)に関する記述を行います.
%%
%%		(a)jbook	本のような長い文書用(日本語対応)
%%		(b)jarticle	学術論文のような比較的短い文書用(日本語対応)
%%		(c)jreport	レポート形式(日本語対応)
%%		(d)jletter	手紙用(日本語対応)
%%
%%	    ◎ スタイルオプション(デフォルトの文字サイズは 10pt)
%%	       スタイルオプションでは,文書形式の詳細なスタイルを記述します.
%%	       例えば,文字の大きさや段組などの指定を行ないます.
%%
%%		(a)a4j	用紙サイズを A4 にする
%%		(b)11pt	文字サイズを 11pt に指定
%%		(c)12pt	文字サイズを 12pt に指定
%%		(d)twoside	両面印刷用に偶数ページと奇数ページの
%%				レイアウトを変える.書籍のような見開きにする
%%				ため,綴じしろを左右のページで変更する.
%%		(e)twocolumn	二段組にする.
%%		(f)titlepage	表題を独立したページに出力する.
%%				(article, jarticle スタイルのみで有効)
%%		(g)openbib	オープン形式の文献リストを生成する.
%%		(h)leqno	数式の番号を左端に出力.(デフォルトでは右端)
%%		(i)fleqn	数式を左に寄せて出力.(デフォルトでは中央揃え)
%%
\documentclass[a4j]{jreport}
\西暦	% \today の表示を「西暦」で出力するように変更(デフォルトは「和暦」)

%%
%%	(2)レイアウトパラメータの変更
%%	    レイアウトパラメータとは,上下左右マージンや改行幅などの
%%	    レイアウトを設定する.
%%
%\setlength{\textheight}{22cm}		% 本文領域の高さ
%\setlength{\textwidth}{15cm}		% 本文領域の横幅
\setlength{\topmargin}{-12.5truemm}	% ページの上部余白 
%\setlength{\oddsidemargin}{0.5cm}	% 奇数(右)ページの左余白
%\setlength{\evensidemargin}{0.0cm}	% 偶数(左)ページの左余白

%%
%%	(3)LaTeX の機能を拡張する各種スタイルファイルの読み込み
%%	    LaTeX に特別な機能などを拡張するために,マクロ定義などが
%%	    あらかじめ定義されたものにスタイルファイルがあります.
%%	    そのスタイルファイルを読み込むことによって文書全体で利用する
%%	    ことができます.
%%
\usepackage{comment}	% comment 環境を使うためのスタイルファイル

%
%	↓丸みのある四角や影付きの四角などを使うためのスタイルファイル
%
\usepackage{fancybox}	% shadowbox, ovalbox 環境を使うためのスタイルファイル
\usepackage{ascmac}	% screen 環境を使うためのスタイルファイル
\usepackage{framed}	% framed 環境を使うためのスタイルファイル

%
%	↓PostScript 形式 や EPS 形式の図を張り付けるためのスタイルファイル
%
\usepackage{float}	% 図や表を記述した位置に置くためのスタイルファイル
\usepackage[dvipdfmx]{graphicx}	% 回転などもできるスタイルファイル

%
%	箇条書きに関するスタイルファイル
%
\usepackage{enumitem}	% 拡張 enumerate 環境を使うためのスタイルファイル

%%
%%	(4)マクロコマンドの登録,コマンドの再定義
%%	    LaTeX では,コマンド列をマクロとして登録すると,それがあたかも
%%	    一つのコマンドであるかのように使用することができます.
%%	    同様の理由で LaTeX 標準のコマンドの再定義を文書全体で有効に
%%	    するときにも,プリアンブル内で再定義を行ないます.
%%

%
%	図や表における説明タイトル(\caption)のデフォルトスタイルの変更
%	したい場合には,ここに,記述することによって変更することができます.
%	例えば,以下に示すようなスタイル変更を行なうことができます.
%
%	● キャプションの変更例 [1]
%
%			 デフォルトスタイル	 変更後のスタイル
%	【図の場合】 	図 1.1:  図のタイトル	図 1:  図のタイトル
%	【表の場合】	表 1.1:  表のタイトル	表 1:  表のタイトル
%
%
%	● キャプションの変更例 [2]
%
%			 デフォルトスタイル	 変更後のスタイル
%	【図の場合 a】	図 1.1:  図のタイトル	図 1.1  図のタイトル
%	【表の場合 a】	表 1.1:  表のタイトル	表 1.1  表のタイトル
%	【図の場合 b】	図 1:    図のタイトル	図 1    図のタイトル
%	【表の場合 b】	表 1:    表のタイトル	表 1    表のタイトル
%
%	これらの,具体的な変更方法については,e-Learning システム
%
%		http://phoenix.info.nara-k.ac.jp/moodle/
%
%	において,サンプルを示してありますので参考にして下さい.
%

%%
%%	(5)表題用のパラメータ設定
%%	    LaTeX は表題の体裁を整えて出力できます.
%%	    表題には,文書のタイトル,著者名,日付などがあります.
%%
%%	    表題(表紙)の出力方法は,以下に示す 3 通りがあります.
%%
%%		(A) プリアンブル部で記述する場合
%%
%%			\title{\textbf{コンピュータネットワークに関する実験}}
%%			\author{\large 西野 貴之(nishino@info.nara-k.ac.jp)}
%%			\date{平成 28 年度(4 学年用)\\
%%			\vspace*{18.0cm}
%%			\textbf{奈良工業高等専門学校 情報工学科}}
%%
%%		    のように記述し,本文領域の先頭(\begin{document} の後)に,
%%		    \maketitle
%%		    を記述することで,タイトルなどが出力されます.
%%
%%		(B) \titlepage 環境を使用する場合
%%
%%		    本文領域に \begin{titlepage} ~ \end{titlepage} を記述し,
%%		    その領域内に表紙内容を記述する.(以下参照)
%%
%%		(C) スタイルファイルを書く
%%

%%
%% ★ 本文領域
%%
%%	ここからが,実際の文章領域です.本文は,必ず
%%
%%	\begin{document}
%%	      :
%%	     本文
%%	      :
%%	\end{document}
%%
%%	の間に記述する必要があります.
%%
\begin{document}	% 本文領域の宣言

%
% titlepage 環境を使った表紙
%
\begin{titlepage}
  \centering		% センタリング命令
  \vfill
  \vspace*{10cm}
  表紙は各自で作成してください.
  \vspace*{10cm}
  \vfill
\end{titlepage}

%\maketitle		% タイトルの作成 (\title{} や \author{} を使う場合)

\chapter*{実験報告書のサンプルについて}
%         ~~~~~~~~~~~~~~~~~~~~~~~~~~~~
%         ↑
%         \LaTeX における,章番号や節番号などを自動的に出力する命令です.
%
%   (1)\chapter{}	   % {} 内の文字に対して自動的に章番号を割当てます.
%   (2)\section{}	   % {} 内の文字に対して自動的に節番号を割当てます.
%   (3)\subsection{}	   % {} 内の文字に対して自動的に小節番号を割当てます.
%   (4)\subsubsection{}  % {} 内の文字に対して自動的に小小節番号を割当てます.
%
% これらの命令は,本文中で上記の命令が現れるたびに,章,節,小節,小小節番号
% を自動的にインクリメントします.
% それに対して,以下の命令では,{} 内の文字に対して,番号を割り当てずその文字
% だけを出力します.
%
%   (a)\chapter*{}
%   (b)\section*{}
%   (c)\subsection*{}
%   (d)\subsubsection*{}
%
情報工学実験 III 「コンピュータネットワークに関する実験」の
実験報告書は,\LaTeX で作成してください.\\
%                                        ~~
%                                        ↑
%                                     改行命令
%
% LaTeX では,本文中における改行は出力には反映されません.
% 意図的な改行したい場合には,このように改行命令を入力する必要があります.
%
% 1 行以上の空白行やあるいは,\chapter, \section などの命令があるとその
% 次の文章(段落)が自動的に全角 1 字分の字下げが自動的に行われます.
%

\LaTeX は,書籍や論文など文書の体裁を整えるためのソフトウェアですから,
あなたは文書を書くことに集中するだけで基本的なレイアウトは \LaTeX が
整えてくれます.\\

この実験報告書のサンプルでは,実験報告書作成に必要となる図の挿入方法や
表の作成など \LaTeX の複雑な命令については,
できるだけ \textbf{\emph{コメント付き解説}} を行っていますので,
このソースや参考文献\cite{LaTeX2e}を参考にして実験報告書を書いて下さい.\\
%                   ~~~~~~~~~~~~~~
%                         ↑
%                  参考文献の項目に記述したラベルを参照するための命令です.

この実験報告書のソース中で ``\texttt{\%}'' 記号で始まる行や,
``\texttt{\%}'' 以降行末まではコメントとして扱われます.
% ~~~~~~~~~~~                      
% タイプライタ風フォント
%	{} 中の文字をタイプライタフォントにします.
%	LaTeX では '%' 記号は「コメント」として解釈します.
%	'%' 記号の前に '\' 記号を付けることにより,なる文字として解釈
%	されるようにしています.
%	これを「エスケープする」といいます.
%
コメントは \LaTeX のコンパイル時に無視され印刷イメージには反映されません.\\

最後に,初めて \LaTeX を使う人は,\LaTeX の操作はもちろん,
\LaTeX やその関連ツールのインストールについても苦労すると思いますので,
早めにとりかかるようにしてください.

%
%	右寄せ出力する環境
%
\begin{flushright}
平成 19 年 4 月 18 日(水)\\
西野 貴之(nishino@info.nara-k.ac.jp)
\end{flushright}


\chapter*{実験結果の記述ポイント}

ここでは,情報工学実験 III 「コンピュータネットワークに関する実験」
実験報告書における実験結果の記述例を示します.\\

皆さんが,このサンプルの実験報告書を読んで,これらの課題を再現でき,
かつその実行結果を理解することができるかを確認して下さい.

もし,これらのサンプル課題が再現出来なかったり,
理解もできないのであれば,このサンプルの実験報告内容では,
「\textgt{不十分}」であるということです.\\	% \textgt はゴシック体

そして,皆さんが提出する実験報告書では,そのようなことがないように
記述するようにして下さい.

%
%	右寄せ出力する環境
%
\begin{flushright}
2016 年 5 月 27 日(金)\\
西野 貴之(nishino@info.nara-k.ac.jp)
\end{flushright}

\chapter*{アブストラクト}


% \chapter{実験の目的}		% 平成 28 年度は不要

%%
%%
%%
\chapter{サンプル課題}
\section{基本コマンド編}

\begin{enumerate}[labelindent=\parindent, leftmargin=*, label=課題 \arabic*)]
%                ~~~~~~~~~~~~~~~~~~~~~~~~~~~~~~~~~~~~~~~~~~~~~~~~~~~~~~~~~~~~~
%                 拡張 enumerate 環境による箇条書きの書式指定
%
% 箇条書きの書き方
%
%	箇条書を行うには,以下に示す 4 通りの方法があります.
%
%	(1)記号による箇条書き
%		\begin{itemize}
%		  \item 項目-A		出力	・ 項目-A
%		  \item 項目-B		 →	・ 項目-B
%		\end{itemize}
%	(2)数字による箇条書き
%		\begin{enumerate}
%		  \item 項目-A		出力	1. 項目-A
%		  \item 項目-B		 →	2. 項目-B
%		\end{enumerate}
%	(3)数字による箇条書き(enumerate 環境の拡張機能)
%		\begin{enumerate}[(a)]
%		  \item 項目-A		出力	(a)項目-A
%		  \item 項目-B		 →	(b)項目-B
%		\end{enumerate}
%	(4)ユーザ定義による箇条書き
%		\begin{description}
%		  \item [ラベル] 項目-A		出力	[ラベル] 項目-A
%		  \item [ラベル] 項目-B		 →	[ラベル] 項目-B
%		\end{description}
%
  \item 1752 年 9 月のカレンダーを示せ.\\

	FreeBSD では,\texttt{cal} コマンドに \textit{month},
	\textit{year} の順で引数を指定することにより任意の年月の
	カレンダーを表示することができる.

	コマンドラインと実行結果を図 \ref{Figure: cal 9 1752} に示す.

	  %
	  % 端末画面風の図を出力するための宣言(\begin{figure} ~ \end{figure})
	  %
	  \begin{figure}[H]	% [H] はこの記述場所に表を挿入するため記述子
	    \centering		% センタリング命令
              \begin{screen}[3]	% [3] は,screen 環境で囲み枠カーブの丸み設定
                \setlength{\baselineskip} {4mm} % {screen} 環境内の行間を 4mm に指定する.
		% ↓ \begin{verbatim} はプログラムリストなどを出力する場合に
		%    用いるもので,\end{verbatim} まで囲まれた部分の記述を
		%    そのまま出力します.
		%    LaTeX では,'%' はコメントとして解釈されるが verbatim
		%    環境内では,単なる文字として扱われる.
	        \begin{verbatim}
	nishino@sleipnir<~>[1]% cal 9 1752
	      9月 1752
	日 月 火 水 木 金 土
	       1  2 14 15 16
	17 18 19 20 21 22 23
	24 25 26 27 28 29 30
	nishino@sleipnir<~>[2]%
	        \end{verbatim}
	        \vspace*{-18pt}
	%       ~~~~~~~~~~~~~~
	%       ↑丸みのある四角 {screen} とプロンプト(nishino@sleipnir<~>[1]%)
	%         の行間調整
	      \end{screen}
	%     \vspace*{-0.9cm}
	%     ~~~~~~~~~~~~~~~~
	%     ↑丸みのある四角 {screen} と {caption} の行間の空間調整
	      \caption{1752 年 9 月のカレンダー表示結果} % 図の説明タイトル
	      \label{Figure: cal 9 1752} % 図を本文中から参照する時のラベル
	  \end{figure}

	図 \ref{Figure: cal 9 1752} において,1752 年 9 月 2 日の
	翌日が 9 月 14 日になっているのは,\texttt{jman cal}(オンライン
	マニュアル)によると,$\cdots$ のためである.

	オンラインマニュアルのコマンドラインと実行結果(引用部分)を
	図 \ref{Figure: jman cal} に示す.

	  \begin{figure}[H]
	    \centering
	      \begin{screen}[3]
	        \setlength{\baselineskip} {4mm}
	        \begin{verbatim}
	nishino@sleipnir<~>[2]% jman cal
	             :
	             :
	             :
	nishino@sleipnir<~>[3]%
	        \end{verbatim}
	        \vspace*{-18pt}
	      \end{screen}
	%     \vspace*{-0.9cm}
	      \caption{cal コマンドのオンラインマニュアル}
	      \label{Figure: jman cal}
	  \end{figure}

  \item ホームディレクトリにある 4IExp.pdf ファイルを各圧縮コマンド
	(\texttt{bzip2, compress, gzip, xz, zip})で圧縮した場合の
	圧縮比率と実行時間を示せ.\\

%	※ この課題はサンプルなので,実験システムのホームディレクトリ
%	   に .txt ファイルはありません.
%	   この課題を模擬的に行う場合には,適当なファイルを各自で用意
%	   (作成あるいはコピーするなど)して下さい.
%	   また,xz コマンドなどシステムによってはインストール
%	   されていない場合があります.

	ファイル圧縮を行う前に,
	\texttt{zip} コマンドを除く \texttt{compress} などの圧縮コマンドは,
	圧縮すると圧縮前のファイルを置き換えてしまうため,あらかじめ
	\texttt{cp} コマンドで 4IExp.pdf ファイルの複製を行った.

	複製のコマンドラインと実行結果を図 \ref{Figure: copy of file} に示す.

	  \begin{figure}[H]
	    \centering
	      \begin{screen}[3]
	        \setlength{\baselineskip} {4mm}
	        \begin{verbatim}
	nishino@sleipnir<~>[11]% ls -l 4IExp*     # 4IExp のファイルの存在確認
	-rw-r--r--  1 nishino  staff    2608544  May  28 23:58  4IExp.pdf
	nishino@sleipnir<~>[12]% cp 4IExp.pdf 4IExp_bzip2
	nishino@sleipnir<~>[13]% cp 4IExp.pdf 4IExp_compress
	nishino@sleipnir<~>[14]% cp 4IExp.pdf 4IExp_gzip
	nishino@sleipnir<~>[15]% cp 4IExp.pdf 4IExp_xz
	nishino@sleipnir<~>[16]% cp 4IExp.pdf 4IExp_zip
	nishino@sleipnir<~>[17]% ls -l 4IExp*     # 以下,圧縮前の各ファイルサイズ
	-rw-r--r--  1 nishino  staff    2608544  May  28 23:58  4IExp.pdf
	-rw-r--r--  1 nishino  staff    2608544  May  28 23:58  4IExp_bzip2
	-rw-r--r--  1 nishino  staff    2608544  May  28 23:58  4IExp_compress
	-rw-r--r--  1 nishino  staff    2608544  May  28 23:58  4IExp_gzip
	-rw-r--r--  1 nishino  staff    2608544  May  28 23:58  4IExp_xz
	-rw-r--r--  1 nishino  staff    2608544  May  28 23:58  4IExp_zip
	nishino@sleipnir<~>[18]%
	        \end{verbatim}
	        \vspace*{-18pt}
	      \end{screen}
	%     \vspace*{-0.9cm}
	      \caption{4IExp.pdf ファイルの複製}
	      \label{Figure: copy of file}
	  \end{figure}

	コマンドの実行時間の計測には,
	\texttt{/usr/bin/time} コマンドを使用した.
	また,各圧縮コマンドでは,
	圧縮比率など詳細な情報を表示する \texttt{-v} オプションを指定した.

	各圧縮コマンドによるファイル圧縮および実行時間の計測を
	図 \ref{Figure: /usr/bin/time bzip2, compress, gzip, xz, zip} に示す.

	  \begin{figure}[H]
	    \centering
	      \begin{screen}[3]
	        \setlength{\baselineskip} {4mm}
		%{\scriptsize
	        \begin{verbatim}
	nishino@sleipnir<~>[18]% /usr/bin/time bzip2 -v 4IExp_bzip2
	  4IExp_bzip2:  2.197:1,  3.641 bits/byte, 54.49% saved, 2608544 in, 1187149 out.
		0.59 real         0.45 user         0.12 sys
	nishino@sleipnir<~>[19]% /usr/bin/time compress -v 4IExp_compress
	4IExp_compress.Z: 64% compression
		0.17 real         0.11 user         0.04 sys
	nishino@sleipnir<~>[20]% /usr/bin/time gzip -v 4IExp_gzip
	4IExp_gzip:         52.2% -- replaced with 4IExp_gzip.gz
		0.79 real         0.72 user         0.06 sys
	nishino@sleipnir<~>[21]% /usr/bin/time xz -v 4IExp_xz
	4IExp_xz (1/1)
	  100 %     986.2 KiB / 2,547.4 KiB = 0.387                   0:02
		2.44 real         2.05 user         0.32 sys
	nishino@sleipnir<~>[22]% /usr/bin/time zip -v 4IExp_zip.zip 4IExp_zip
	  adding: 4IExp_zip      (in=2608544) (out=1248500) (deflated 52%)
	total bytes=2608544, compressed=1248500 -> 52% savings
		0.34 real         0.29 user         0.02 sys
	nishino@sleipnir<~>[23]% ls -l 4IExp*    # 以下,圧縮後の各ファイルサイズ
	-rw-r--r--  1 nishino  staff    2608544  May  28 23:58  4IExp.pdf
	-rw-r--r--  1 nishino  staff    1187149  May  28 23:58  4IExp_bzip2.bz2
	-rw-r--r--  1 nishino  staff    1662727  May  28 23:58  4IExp_compress.Z
	-rw-r--r--  1 nishino  staff    1245941  May  28 23:58  4IExp_gzip.gz
	-rw-r--r--  1 nishino  staff    1009920  May  28 23:58  4IExp_xz.xz
	-rw-r--r--  1 nishino  staff    1248666  May  29 00:01  4IExp_zip.zip
	nishino@sleipnir<~>[24]%
	        \end{verbatim}%}
	        \vspace*{-18pt}
	      \end{screen}
	%     \vspace*{-0.9cm}
	      \caption{各圧縮コマンドによるファイル圧縮と実行時間の計測}
	      \label{Figure: /usr/bin/time bzip2, compress, gzip, xz, zip}
	  \end{figure}

	図 \ref{Figure: /usr/bin/time bzip2, compress, gzip, xz, zip} に
	おいて,各圧縮コマンドの \texttt{-v} オプションは,
	圧縮比率の計算に違い(圧縮比率と削減比率)があることから,
	圧縮比率は式(\ref{Equation: Compression Ratio})により算出した.
	また,\texttt{/usr/bin/time} コマンドは,\texttt{real} が引数で
	指定されたコマンドの実行に要した時間[秒]である.

	\begin{equation}
	  圧縮比率 [\%] = \frac{圧縮後のファイルサイズ [byte] }
	  			{圧縮前のファイルサイズ [byte] } \times 100
				\label{Equation: Compression Ratio}
	\end{equation}


	図 \ref{Figure: /usr/bin/time bzip2, compress, gzip, xz, zip} および
	式(\ref{Equation: Compression Ratio})より,
	各圧縮コマンドの圧縮比率と実行時間を
	表 \ref{Table: CompressionRatio and Times} と
	図 \ref{Figure: CompressionRatio and Times} に示す.

	%
	%	表の宣言(\begin{table} ~ \end{table})
	%
	\begin{table}[H]
	  \caption{各圧縮コマンドの圧縮比率と実行時間}	% 表の説明タイトル
	  \label{Table: CompressionRatio and Times}	% 本文中からの参照ラベル
	  \centering	% センタリング命令
	    \vspace{0.2cm}
	    \begin{tabular}{c|c|c|c|c|c}
%		            ~~~~~~~~~~~
%			    ↑この部分が実際の表の列の定義部分になります.
%
%		この場合,6 つの列を定義しています.
%		'c' は,セル内の文字位置の指定でセンタリングを表しており,
%			セル幅は.文字数に応じて自動的に調整されます.
%		'|' 記号は,縦の罫線を引くことを表しています.
%			この場合,表の両端は罫線を引かず,セル間だけ罫線を
%			引くことになります.
%
%		もし,セル幅を指定する場合には,以下のように記述してください.
%
%		[例] \begin{tabular}{p{2.0cm}|p{2.0cm}|p{5.0cm}}  % 3 列の場合
%
		\hline
			&
			圧縮比率	& 実行時間	&
			\multicolumn{2}{c|}{ファイルサイズ[byte]} &	\\
%		~~~~~~~~~~~~~~~~~~~~~~~~~~~~~~~~~~~~~~~~~~~~~~~~~~~~~~~~~~
%		 ↑ セル内の内容を記述します.
%		    セルとセルの区切りには,"&" 記号を用い,末尾のセルには,
%		    "\\" を記述し,最終行(セル)であることを表します.
%		    上記のことが守られていれば,改行位置は任意です.
%
		\cline{4-5}
		\raisebox{1.5ex}[0pt]{圧縮コマンド}	&
			[\%]	& [秒]	&
			~~~圧縮後~~~	& ~~~圧縮前~~~	&
			\raisebox{1.5ex}[0pt]{拡張子}	\\
		\hline
		\texttt{compress} &
			63.74 & 0.17 & 1,662,727 & 	     & \texttt{.Z}   \\
		\texttt{zip}	&
			47.87 & 0.34 & 1,248,666 & 	     & \texttt{.zip} \\
		\texttt{gzip}	&
			47.76 & 0.79 & 1,245,941 & 2,608,544 & \texttt{.gz}  \\
		\texttt{bzip2}	&
			45.51 & 0.59 & 1,187,149 & 	     & \texttt{.bz2} \\
		\texttt{xz}	&
			38.72 & 2.44 & 1,009,920 & 	     & \texttt{.xz}  \\
		\hline
		\hline
		平均	& 48.72 & 0.87 & \multicolumn{3}{c}{}		\\
		\hline
	    \end{tabular}
	\end{table}

	\begin{figure}[H]
	  \centering
	  \includegraphics[clip, scale=0.8]{epsf/compression_ratio.eps}
	  \vspace*{-0.5cm}
	  \caption{各圧縮コマンドの圧縮比率と実行時間}
	  \label{Figure: CompressionRatio and Times}
	\end{figure}

%	↓上記なら,こんな感じでしょうか ……
	図 \ref{Figure: CompressionRatio and Times} より,
	4IExp.pdf ファイルの圧縮においては,
	最近 UNIX で使用される \texttt{xz} コマンドの圧縮率が最も高く,
	伝統的な UNIX の圧縮 コマンドである \texttt{compress} は,
	平均より圧縮率は低く,実行時間は短かった.

	\texttt{bzip2},\texttt{gzip},\texttt{zip} コマンドは,
	圧縮率にあまり差は見られなかった.その中でも \texttt{zip} コマンドは,
	実行時間が短く書庫である ZIP ファイルは様々なプラットフォームで
	圧縮・解凍ができるなどの特徴があることから広く使われている.
											しかし,書庫の互換性やアルゴリズムなど $\cdots$

\end{enumerate}


\newpage				% 改ページ
\section{ネットワーク基本コマンド編}
\begin{enumerate}[labelindent=\parindent, leftmargin=*, label=課題 \arabic*)]
  \item \texttt{host} コマンドを使用して情報工学科の公式 Web サーバ
	\texttt{www.info.nara-k.ac.jp} の IP アドレスを示せ.\\

	\texttt{host} コマンドは,引数に指定された形式
	(ホスト名もしくは IP アドレス)の逆を検索する.
	また,コマンドオプションを省略すると様々な情報が表示されるため,
	「ホスト名(FQDN)」から「IP アドレス」を
	検索する \texttt{-t a} オプションを指定した.

	コマンドラインと実行結果を
	図 \ref{Figure: host www.info.nara-k.ac.jp} に示す.

	  \begin{figure}[H]
	    \centering
	      \begin{screen}[3]
	        \setlength{\baselineskip} {4mm}
	        \begin{verbatim}
	nishino@sleipnir<~>[1]% host -t a www.info.nara-k.ac.jp
	www.info.nara-k.ac.jp is an alias for zeus.info.nara-k.ac.jp.
	zeus.info.nara-k.ac.jp has address 202.24.246.1
	nishino@sleipnir<~>[2]%
	        \end{verbatim}
	        \vspace*{-18pt}
	      \end{screen}
	%     \vspace*{-0.9cm}
	      \caption{情報工学科 公式 Web サーバの IP アドレス}
	      \label{Figure: host www.info.nara-k.ac.jp}
	  \end{figure}

	図 \ref{Figure: host www.info.nara-k.ac.jp} より,
	情報工学科の公式 Web サーバ \texttt{www.info.nara-k.ac.jp} は,

	\noindent
	\texttt{zeus.info.nara-k.ac.jp} の別名(alias)であることが分かる.

	従って,情報工学科の公式 Web サーバの IP アドレスは,
	\texttt{202.24.246.1} である.
\end{enumerate}



\newpage				% 改ページ
\section{システム設定編}
\begin{enumerate}[labelindent=\parindent, leftmargin=*, label=課題 \arabic*)]
  \item \texttt{chsh} コマンドでログインシェルに \texttt{zsh} を選択
	できるように \texttt{/etc/shells} ファイルを設定せよ.
	(インストールされていない場合には,インストールも行うこと.)\\

	まず最初に,\texttt{which} コマンドを使って \texttt{zsh} が
	インストールされているかを調べる.
	\texttt{which} コマンドは,引数で指定されたコマンドを環境変数 PATH
	に設定されているディレクトリ順に検索し,絶対パスで表示する.

	\texttt{which} コマンドの実行結果を図 \ref{Figure: which zsh} に示す

	  \begin{figure}[H]
	    \centering
              \begin{screen}[3]
                \setlength{\baselineskip} {4mm}
	        \begin{verbatim}
	nishino@sleipnir<~>[1]% which zsh
	/usr/local/bin/zsh
	nishino@sleipnir<~>[2]%
	        \end{verbatim}
	        \vspace*{-18pt}
	      \end{screen}
	%     \vspace*{-0.9cm}
	      \caption{which コマンドによる zsh の検索}
	      \label{Figure: which zsh}
	  \end{figure}

	図 \ref{Figure: which zsh} より,\texttt{zsh} は~
	\texttt{/usr/local/bin} にインストールされていることが確認できた.\\

	\texttt{/etc/shells} は,ログインシェルとして使用可能なリストを
	絶対パスで記述したファイルで,\texttt{/usr/local/bin/zsh} を~
	\texttt{/etc/shells} ファイルの末尾に追加した.
	%
	% FreeBSD の場合,シェルをインストールすると自動的に追加してくれます.
	% あくまでもサンプル課題ってことで ……
	%

	編集後の \texttt{/etc/shells} ファイルを
	図 \ref{Figure: 編集後の /etc/shells ファイル} に示す.

	\begin{figure}[H]
	  \centering
	    \begin{framed}
	      \setlength{\baselineskip} {4mm}
	      \begin{verbatim}
	#
	# $FreeBSD: releng/9.3/etc/shells 59717 2000-04-27 21:58:46Z ache $
	#
	# List of acceptable shells for chpass(1).
	# Ftpd will not allow users to connect who are not using
	# one of these shells.

	/bin/sh
	/bin/csh
	/bin/tcsh
	/usr/local/bin/bash
	/usr/local/bin/zsh                                 # この行を追加
	      \end{verbatim}
	      \vspace*{-18pt}
	    \end{framed}
	    \vspace*{-0.3cm}
	    \caption{編集後の /etc/shells ファイル}
	    \label{Figure: 編集後の /etc/shells ファイル}
	\end{figure}

	図 \ref{Figure: chsh コマンドによるログインシェルの変更} に
	\texttt{chsh} コマンドによるログインシェルの変更を示す.

	  \begin{figure}[H]
	    \centering
	      \begin{screen}[3]
	        \setlength{\baselineskip} {4mm}
	        \begin{verbatim}
	nishino@sleipnir<~>[3]% chsh -s /usr/local/bin/zsh
	Password: **********           # パスワードは何も表示されない
	chsh: user information updated
	nishino@sleipnir<~>[4]%
	        \end{verbatim}
	        \vspace*{-18pt}
	      \end{screen}
	%     \vspace*{-0.9cm}
	      \caption{chsh コマンドによるログインシェルの変更}
	      \label{Figure: chsh コマンドによるログインシェルの変更}
	  \end{figure}

	ログインシェルの変更は,
	図 \ref{Figure: chsh コマンドによるログインシェルの変更} に示すように,
	\texttt{chsh} コマンドに変更するログインシェルを引数で与えることが
	できる \texttt{-s} オプションを用いて変更した.

\end{enumerate}

%	
% * 実験課題
%
\chapter{実験課題}

%%
%%
%%
\section{UNIX の基礎}


%%
%%
%%
\newpage				% 改ページ
\section{管理者への第 1 歩}

第 2 週目の図

\begin{figure}[H]
  \centering
  \includegraphics[clip, scale=0.8]{epsf/exam_enviroment.eps}
  \vspace*{-0.3cm}
  \caption{環境設定ファイル群のコピー}
  \label{Figure: 環境設定ファイル群のコピー}
\end{figure}


%%
%%
%%
\newpage				% 改ページ
\section{管理者への第 2 歩}

第 3 週目の図\\

図 \ref{Figure: 実験用ネットワーク(1)} に示すネットワークを構築せよ.

\begin{figure}[H]
  \centering
  \includegraphics[clip, scale=1.1]{epsf/3rd_week_exam_network.eps}
  \vspace*{-0.3cm}
  \caption{実験用ネットワーク(1)}
  \label{Figure: 実験用ネットワーク(1)}
\end{figure}


%%
%%
%%
\newpage	% 改ページ
\section{ネットワークを使ってのファイル転送}

第 4 週目の課題 1 と 課題 2 の図表\\

図 \ref{Figure: ftp によるファイル転送の実験(1)} に
示すネットワークを用いて,ftp コマンドを用いてファイル転送を行い,
各マシン間の転送時間と転送速度を調べよ.

\begin{figure}[H]
  \centering
  \includegraphics[clip, scale=1.1]{epsf/4th_week_ftp_exam12.eps}
  \vspace*{-0.3cm}
  \caption{ftp によるファイル転送の実験(1)}
  \label{Figure: ftp によるファイル転送の実験(1)}
\end{figure}

\begin{table}[H]
  \caption{sleipnir とリピータ HUB によるファイル転送(1 台)}
  \label{Table: sleipnir とリピータ HUB によるファイル転送(1 台)}
% \vspace{-0.2cm}
  \centering
  \begin{tabular}{|p{1cm}|p{2.5cm}|p{2.5cm}|p{3.5cm}|p{3cm}|}
	\hline
	\multicolumn{1}{|c|}{台数}		&
%	~~~~~~~~~~~~~*~~===~~++++~
%	この \multicolumn{1}{|c|}{台数} の記述は,セルの結合命令で,
%	この場合は,結合を行なっていませんが,左端のセルに対して
%	"台数" という文字列のセンタリングを行ない,かつ,セルの
%	左右には縦の罫線を引くことを指定しています.(以下同様)
%	
		\multicolumn{1}{c|}{転送ホスト} &
		\multicolumn{1}{c|}{転送時間}	&
		\multicolumn{1}{c|}{転送速度}	&
		\multicolumn{1}{c|}{備考}	\\
	\hline
	\multicolumn{1}{|c|}{1 台}		&
		\multicolumn{1}{c|}{mint01 $\leftarrow$ sleipnir} &
			&			% 転送時間欄
			&			% 転送速度欄
		\multicolumn{1}{c|}{Binary mode, get}	\\	% 備考欄
	\hline
  \end{tabular}
\end{table}

\begin{table}[H]
  \caption{sleipnir とリピータ HUB によるファイル転送(2 台)}
  \label{Table: sleipnir とリピータ HUB によるファイル転送(2 台)}
% \vspace{-0.2cm}
  \centering
  \begin{tabular}{|p{1.0cm}|p{2.5cm}|p{2.5cm}|p{3.5cm}|p{3.0cm}|}
	\hline
	\multicolumn{1}{|c|}{台数}		&
		\multicolumn{1}{c|}{転送ホスト} &
		\multicolumn{1}{c|}{転送時間}	&
		\multicolumn{1}{c|}{転送速度}	&
		\multicolumn{1}{c|}{備考}	\\
	\hline
		& \multicolumn{1}{c|}{mint01 $\leftarrow$ sleipnir} &
			&		% 転送時間欄
			&		% 転送速度欄
			\\		% 備考欄
	\cline{2-4}
	\multicolumn{1}{|c|}{\raisebox{1.6ex}[0pt]{2 台}} &
%			     ^^^^^^^^^^^^^^^^^^^^^^^^^^^
%			     この部分は 2 行のセルの中間に "2 台" という
%			     文字が出力するように,\raisebox コマンドを
%			     使って,文字列を半行上に持ち上げています.
%
		\multicolumn{1}{c|}{mint02 $\leftarrow$ sleipnir} &
			&		% 転送時間欄
			&		% 転送速度欄
		\multicolumn{1}{c|}{\raisebox{1.6ex}[0pt]{Binary mode, get}} \\
	\hline
	\multicolumn{2}{|c|}{平均}	&
			&		% 転送時間欄
			&		% 転送速度欄
			\\		% 備考欄
	\hline
  \end{tabular}
\end{table}


第 4 週目の課題 3 と 課題 4 の図\\

図 \ref{Figure: ftp によるファイル転送の実験(2)} に示す
ネットワークを用いて,ftp コマンドを用いてファイル転送を行い,
各マシン間の転送時間と転送速度を調べよ.

\begin{figure}[H]
  \centering
  \includegraphics[clip, scale=1.1]{epsf/4th_week_ftp_exam34.eps}
  \vspace*{-0.3cm}
  \caption{ftp によるファイル転送の実験(2)}
  \label{Figure: ftp によるファイル転送の実験(2)}
\end{figure}


%%
%%
%%
\newpage				% 改ページ
\section{今までの総復習}

第 5 週目の図\\

図 \ref{Figure: 実験用ネットワーク(2)} に示すネットワークを構築せよ.

\begin{figure}[H]
  \centering
  \includegraphics[clip, scale=1.1]{epsf/5th_week_exam_network.eps}
  \vspace*{-0.3cm}
  \caption{実験用ネットワーク(2)}
  \label{Figure: 実験用ネットワーク(2)}
\end{figure}


\chapter{考察}

%
% * 参考文献
%
%
\begin{thebibliography}{99}
  \bibitem{LaTeX2e}
	奥村 晴彦,						% 著者名
	\LaTeX2e 美文書作成入門,				% 書名
	第 3 版,						% 版表示
	東京,							% 出版地
	技術評論社,						% 出版者(社)
	平成 17 年,						% 出版年
	403 頁							% 総ページ数

  \bibitem{Absolute FreeBSD}
	Michael Lucas,						% 著者名
	Absolute BSD FreeBSD システム管理とチューニング,	% 書名
	佐藤 広生 監訳,					% 翻訳者名
	毎日コミュニケーションズ,				% 出版者(社)
	2004 年,						% 出版年
	685 頁							% 総ページ数

  \bibitem{How to write references}
	科学技術振興機構,					% 著者名
	参考文献の役割と書き方 科学技術情報流通技術基準(SIST)の活用,% Web サイトのタイトル
	2011 年,						% 出版年
	23 頁,
	https://jipsti.jst.go.jp/sist/pdf/SIST\_booklet2011.pdf,% Web サイトの URL
	(参照 2016-May-26)					% 入手日付

\end{thebibliography}


\chapter{実験指導書の訂正}


\chapter{実験に対する意見や感想}


\end{document}
